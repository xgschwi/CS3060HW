\documentclass[paper=letter, fontsize=11pt]{scrartcl} % A4 paper and 11pt font size

\usepackage{enumitem}
\usepackage{listings,multicol}
\usepackage[T1]{fontenc} % Use 8-bit encoding that has 256 glyphs
\usepackage{fourier} % Use the Adobe Utopia font for the document - comment this line to return to the LaTeX default
\usepackage[english]{babel} % English language/hyphenation
\usepackage{amsmath,amsfonts,amsthm} % Math packages
\usepackage{lipsum} % Used for inserting dummy 'Lorem ipsum' text into the template
\usepackage{sectsty} % Allows customizing section commands
\allsectionsfont{\centering \normalfont\scshape} % Make all sections centered, the default font and small caps
\usepackage{fancyhdr} % Custom headers and footers
\pagestyle{fancyplain} % Makes all pages in the document conform to the custom headers and footers
\fancyhead{} % No page header - if you want one, create it in the same way as the footers below
\fancyfoot[L]{} % Empty left footer
\fancyfoot[C]{} % Empty center footer
% \fancyfoot[R]{\thepage} % Page numbering for right footer
\renewcommand{\headrulewidth}{0pt} % Remove header underlines
\renewcommand{\footrulewidth}{0pt} % Remove footer underlines
\setlength{\headheight}{13.6pt} % Customize the height of the header

\setlength\parindent{0pt} % Removes all indentation from paragraphs - comment this line for an assignment with lots of text

\usepackage[margin=0.75in]{geometry}
\usepackage{hyperref}
%----------------------------------------------------------------------------------------
%   TITLE SECTION
%----------------------------------------------------------------------------------------

\newcommand{\horrule}[1]{\rule{\linewidth}{#1}} % Create horizontal rule command with 1 argument of height

\title{ 
    \normalfont \normalsize 
    \textsc{CS 3060 Programming Languages, Fall 2020} \\ [25pt] % Your university, school and/or department name(s)
    \horrule{0.5pt} \\[0.4cm] % Thin top horizontal rule
    \huge Assignment \#4  \\ % The assignment title
    \horrule{2pt} \\[0.5cm] % Thick bottom horizontal rule
}

% \author{John Smith} % Your name

% \date{\normalsize\today} % Today's date or a custom date

\begin{document}

    \begin{center}
         Assignment \#4\\
        \small CS 3060 Programming Languages, Fall 2020 \\
        \small Instructor: S. Roy \\
        \huge Prolog \#2
    \end{center}
    
    \textbf{Due Date:} Oct 12 @ 11.59 PM. \\
    \textbf{Total Points:} 60 points \\


    \textbf{Directions:} Using the source provided via Gitlab \@ \texttt{https://gitlab.com/sanroy/fa20-cs3060-hw/},
complete the assignment below. The process for completing this assignment should be as follows:

    \begin{enumerate}[noitemsep]
        \item You already forked the Repository ``sanroy/fa20-cs3060-hw'' to a repository ``yourId/fa20-cs3060-hw'' under your username. If not, do it now.
        \item Get a copy of hw4 folder in ``sanroy/fa20-cs3060-hw'' repository as a hw4 folder in your repository ``yourId/fa20-cs3060-hw''
        \item Complete the assignment, committing changes to git. Each task code should be in a separate file. As an example, task1.pl for Task 1.
        \item Push all commits to your Gitlab repository
        \item If you have done yet done so, add TA and Roy as member of your Gitlab repository
    \end{enumerate}


    \textbf{Tasks:}
    \begin{enumerate}


        \item \textbf{Task \#1:)} (20 points) Consider the following knowledge base as in the previous assignment:

        \texttt{hasDirectFlight(newOrleans, chicago).\\
        hasDirectFlight(philadelphia, newOrleans).\\
        hasDirectFlight(columbus, philadelphia).\\
        hasDirectFlight(sanFrancisco, columbus).\\
        hasDirectFlight(detroit, sanFrancisco).\\
        hasDirectFlight(toledo, detroit).\\
        hasDirectFlight(houston, sanFrancisco).\\
        } \\

        You already wrote a recursive rule \texttt{hasFlightRoute/2} that tells us whether we can travel by plane from one town A to another town B. 
This time you write a recursive rule \texttt{findFlightRoute/2} that should find the route (with zero or more airports as intermediate hops) 
from town A to town B. Your rule not only finds the route, it also prints the route 
(showing each hop seperated by a dash or so) on the computer screen. 
If any such route does not exist, prolog should respond "false" or "no" (as appropriate in your Prolog dialect). 
Run at least 2 querries and show the output (in README) you get from Prolog. \texttt{Writing README carries 2 point.} 

\item \textbf{Task \#2a:)} (10 points) Write a Prolog rule \texttt{deleteFirst(A, List1, List2)} 
which deletes the first occurrence of item A from List1 to produce result List2. 
As an example, running query \texttt{deleteFirst(b, [b,c,d,b,c,b], Result).} should give us $Result = [c, d, b, c, b]$. 
Then, extend your Prolog rule so that it asks the user for the name of the output file so 
that it can present the result in that output file. 
Test your implementation of the rule with at least 2 querries and show the results in README. \texttt{Writing README carries 2 points.}

\textbf{Task \#2b:)} (10 points) Write a Prolog rule \texttt{deleteAll(A, List1, List2)} 
which deletes all occurrences of item A from List1 to produce result List2. 
As an example, running query \texttt{deleteAll(b, [b,c,d,b,c,b], Result).} 
should give us $Result = [c, d, c]$. Then, extend your Prolog rule so that it asks the user for the name of the output file so that it can present the result in that output file. Test your implementation of the rule with at least 2 querries and show the results in README. \texttt{Writing README carries 2 points.}

\item \textbf{Task \#3:)} (20 points) Write Prolog code which can solve any given 9x9 Sudoku puzzle. 
Note that the textbook has solution for the 4x4 Sudoku puzzle. You need to extend the textbook's code 
(i.e., sudoku4.pl on Page 104-105) and make it work for the 9x9 puzzle. Test your implementation with at least 
2 querries and show the results in README. \texttt{Writing README carries 1 point.} 
Note: you may need an alternative to library rules fd\_all\_different and fd\_domain 
depending on your prolog system, and an alternative of these rules are available at the end of lecture3 ppt.


    % \vspace{2cm}
    
    \end{enumerate}

\end{document}


