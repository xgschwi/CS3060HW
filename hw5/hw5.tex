\documentclass[paper=letter, fontsize=11pt]{scrartcl} % A4 paper and 11pt font size

\usepackage{enumitem}
\usepackage{listings,multicol}
\usepackage[T1]{fontenc} % Use 8-bit encoding that has 256 glyphs
\usepackage{fourier} % Use the Adobe Utopia font for the document - comment this line to return to the LaTeX default
\usepackage[english]{babel} % English language/hyphenation
\usepackage{amsmath,amsfonts,amsthm} % Math packages
\usepackage{lipsum} % Used for inserting dummy 'Lorem ipsum' text into the template
\usepackage{sectsty} % Allows customizing section commands
\allsectionsfont{\centering \normalfont\scshape} % Make all sections centered, the default font and small caps
\usepackage{fancyhdr} % Custom headers and footers
\pagestyle{fancyplain} % Makes all pages in the document conform to the custom headers and footers
\fancyhead{} % No page header - if you want one, create it in the same way as the footers below
\fancyfoot[L]{} % Empty left footer
\fancyfoot[C]{} % Empty center footer
% \fancyfoot[R]{\thepage} % Page numbering for right footer
\renewcommand{\headrulewidth}{0pt} % Remove header underlines
\renewcommand{\footrulewidth}{0pt} % Remove footer underlines
\setlength{\headheight}{13.6pt} % Customize the height of the header

\setlength\parindent{0pt} % Removes all indentation from paragraphs - comment this line for an assignment with lots of text

\usepackage[margin=0.75in]{geometry}
\usepackage{hyperref}
%----------------------------------------------------------------------------------------
%   TITLE SECTION
%----------------------------------------------------------------------------------------

\newcommand{\horrule}[1]{\rule{\linewidth}{#1}} % Create horizontal rule command with 1 argument of height

\title{ 
    \normalfont \normalsize 
    \textsc{CS 3060 Programming Languages, Fall 2020} \\ [25pt] % Your university, school and/or department name(s)
    \horrule{0.5pt} \\[0.4cm] % Thin top horizontal rule
    \huge Assignment \#5  \\ % The assignment title
    \horrule{2pt} \\[0.5cm] % Thick bottom horizontal rule
}

% \author{John Smith} % Your name

% \date{\normalsize\today} % Today's date or a custom date

\begin{document}

    \begin{center}
         Assignment \#5\\
        \small CS 3060 Programming Languages, Fall 2020 \\
        \small Instructor: S. Roy \\
        \huge Scala \#1
    \end{center}
    
    \textbf{Due Date:} Oct 29 @ 11.59 PM. \\
    \textbf{Total Points:} 60 points \\


    \textbf{Directions:} Using the source provided via Gitlab \@ \texttt{https://gitlab.com/sanroy/fa20-cs3060-hw/},
complete the assignment below. The process for completing this assignment should be as follows:

    \begin{enumerate}[noitemsep]
        \item You already forked the Repository ``sanroy/fa20-cs3060-hw'' to a repository ``yourId/fa20-cs3060-hw'' under your username. If not, do it now.
        \item Get a copy of hw5 folder in ``sanroy/fa20-cs3060-hw'' repository as a hw5 folder in your repository ``yourId/fa20-cs3060-hw''
        \item Complete the assignment, committing changes to git. Each task code should be in a separate file. As an example, task1.scala for Task 1.
        \item Push all commits to your Gitlab repository
        \item If you have done yet done so, add TA (username: prabeshpaudel) as a member of your Gitlab repository
    \end{enumerate}




    \textbf{Tasks:}

      \begin{enumerate}

        \item \textbf{(12 Points) Task \#1:} Write a Scala program 
which asks the user to type 2 lines (e.g., before going to the
next line the user will hit the 'Enter' key, etc.) on keyboard, 
and saves the lines to a file named "myFile.txt".

        \item \textbf{(12 Points) Task \#2:} Write a Scala program which asks 
the user to type the name of a file.
If the file-content (we are not talking about the filename string)
contains ``Ruby'' or ``ruby'', then print ``The file content is not interesting''. 
If the file-content contains ``Scala'' or ``scala'',
then print ``The file content is interesting''. Otherwise, print "The file is meaningless".

        \item \textbf{(12 Points) Task \#3:} Write a Scala program which 
prints the string ``The cube of $a$ is $b$'' 50 times 
while substituting $a$ by numbers from 5 to 54 where $b$ is $a^3$.


 \item \textbf{(24 Points) function \#4:} Go to \texttt{http://www.textfiles.com/stories/} and check that this site
\footnote {Disclaimer: we did not really check whether this website contains any improper story or language.
If you find something improper, please ignore this site and use some other source} hosts multiple stories
while each story is in a textfile. Download a textfile of your choice, which has atleast 1000 words,
and save the file as \texttt{story.txt}. Your program needs to read this file and process it to
collect some statistics. In particular, report the total number of words in the story,
the number of distinct words,
the second-most frequent word and its frequency. Also, find the number of words which start with character $s$. 
\textbf {Hints:} You may use List and Map (or HashMap) data structures as they are available in Scala. You may design a regular expression to define a \em{word}.



    % \vspace{2cm}

    \end{enumerate}

\end{document}


