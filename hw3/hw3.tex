\documentclass[paper=letter, fontsize=11pt]{scrartcl} % A4 paper and 11pt font size

\usepackage{enumitem}
\usepackage{listings,multicol}
\usepackage[T1]{fontenc} % Use 8-bit encoding that has 256 glyphs
\usepackage{fourier} % Use the Adobe Utopia font for the document - comment this line to return to the LaTeX default
\usepackage[english]{babel} % English language/hyphenation
\usepackage{amsmath,amsfonts,amsthm} % Math packages
\usepackage{lipsum} % Used for inserting dummy 'Lorem ipsum' text into the template
\usepackage{sectsty} % Allows customizing section commands
\allsectionsfont{\centering \normalfont\scshape} % Make all sections centered, the default font and small caps
\usepackage{fancyhdr} % Custom headers and footers
\pagestyle{fancyplain} % Makes all pages in the document conform to the custom headers and footers
\fancyhead{} % No page header - if you want one, create it in the same way as the footers below
\fancyfoot[L]{} % Empty left footer
\fancyfoot[C]{} % Empty center footer
% \fancyfoot[R]{\thepage} % Page numbering for right footer
\renewcommand{\headrulewidth}{0pt} % Remove header underlines
\renewcommand{\footrulewidth}{0pt} % Remove footer underlines
\setlength{\headheight}{13.6pt} % Customize the height of the header

\setlength\parindent{0pt} % Removes all indentation from paragraphs - comment this line for an assignment with lots of text

\usepackage[margin=0.75in]{geometry}
\usepackage{hyperref}
%----------------------------------------------------------------------------------------
%   TITLE SECTION
%----------------------------------------------------------------------------------------

\newcommand{\horrule}[1]{\rule{\linewidth}{#1}} % Create horizontal rule command with 1 argument of height

\title{ 
    \normalfont \normalsize 
    \textsc{CS 3060 Programming Languages, Fall 2020} \\ [25pt] % Your university, school and/or department name(s)
    \horrule{0.5pt} \\[0.4cm] % Thin top horizontal rule
    \huge Assignment \#3  \\ % The assignment title
    \horrule{2pt} \\[0.5cm] % Thick bottom horizontal rule
}

% \author{John Smith} % Your name

% \date{\normalsize\today} % Today's date or a custom date

\begin{document}

    \begin{center}
         Assignment \#3\\
        \small CS 3060 Programming Languages, Fall 2020 \\
        \small Instructor: S. Roy \\
        \huge Prolog \#1
    \end{center}
    
    \textbf{Due Date:} Oct 3 @ 11.59 PM. \\
    \textbf{Total Points:} 60 points \\

    \textbf{Directions:} Using the source provided via Gitlab \@ \texttt{https://gitlab.com/sanroy/fa20-cs3060-hw/},
complete the assignment below. The process for completing this assignment should be as follows:

    \begin{enumerate}[noitemsep]
        \item You already forked the Repository ``sanroy/fa20-cs3060-hw'' to a repository ``yourId/fa20-cs3060-hw'' under your username. If not, do it now.
        \item Get a copy of hw3 folder in ``sanroy/fa20-cs3060-hw'' repository as a hw3 folder in your repository ``yourId/fa20-cs3060-hw''
        \item Complete the assignment, committing changes to git. Each task code should be in a separate prolog file. As an example, task1.pl for Task 1.
        \item Push all commits to your Gitlab repository
        \item If you have done yet done so, add TA and Roy as a member (in `Developer' mode) of your Gitlab repository
    \end{enumerate}


    \textbf{Tasks:}
    \begin{enumerate}

        \item \textbf{Task \#1:} (20 points) Create a knowledge base of your choosing. This knowledge base must include 
at least 12 items (including at least 8 facts and 4 rules) and you must come up with at least 6 querries (whereas at 
least two querries will involve facts, at least two querries will involve rules, and at least two queries will have answer 
NO from prolog). Run the querries and show the answers (in readme) you get from Prolog. \texttt{Writing readme carries 2 points.}

       \item \textbf{Task \#2:} (20 points) Write a rule that will find the largest element of a list of integers. 
Run at least 2 querries and show the answers (in readme) you get from Prolog. 
Your rule should work even if there are duplicate elements, postive integers, zero, and negatives integers. \texttt{Writing readme carries 2 points.}


        \item \textbf{Task \#3:)} (20 points) Consider the following knowledge base: 
Interpretation of \emph{hasDirectFlight(x,y)} is that there is a one-way direct flight connection from airport x to y.
         
        \texttt{hasDirectFlight(newOrleans, chicago).\\
        hasDirectFlight(philadelphia, newOrleans).\\
        hasDirectFlight(columbus, philadelphia).\\
        hasDirectFlight(sanFrancisco, columbus).\\
        hasDirectFlight(detroit, sanFrancisco).\\
        hasDirectFlight(toledo, detroit).\\
        hasDirectFlight(houston, sanFrancisco).\\
        } \\
        Write a recursive rule \texttt{hasFlightRoute/2} that tells us whether there is a flight route 
from one town A to another town B. Run at least 3 querries (one with at least 2-hop route, one with one-hop route, one with no route) 
and show the answers (in readme) you get from Prolog. \texttt{Writing readme carries 2 point.} 

    % \vspace{2cm}
    
    \end{enumerate}

\end{document}


